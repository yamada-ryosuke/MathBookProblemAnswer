\documentclass{jsarticle}
\usepackage{amsmath}

\begin{document}

\section{}
./chapter2/inverseを使って解ける。途中経過を見たい場合は適宜Matrix.hppのMatrix::inverse()にoutput()を挿入すればよい。
\[
	\begin{array}{rrrr}
		イ) &
		\left(
			\begin{array}{rrrr}
				4 & 18 & -16 & -3\\
				0 & -1 &   1 &  1\\
				1 &  3 &  -3 &  0\\
				1 &  6 &  -5 & -1
			\end{array}
		\right) &
		ロ) &
		\left(
			\begin{array}{rrrr}
				 -3 & -1 & 1 & -1\\
				 -3 & -1 & 0 &  1\\
				 -4 & -1 & 1 &  0\\
				-10 & -3 & 1 &  1
			\end{array}
		\right)
	\end{array}
\]

\section{}
行列の変形には./chapter2/ElementaryOperationを用いた。計算過程は./chapter2/2.txt参照。\\
変数を$x = (x_1,\ldots,x_n)$として、$\tilde{x}= ^{t}(x, -1)$と書くこととする。また、方程式の拡大行列を$\tilde{A}$と書くこととする。\\
イ)\\
与式$\tilde{A} \tilde{x}$を変形して、
\begin{eqnarray*}
	% 1行目
	\renewcommand{\arraystretch}{2}
	\left(
		\begin{array}{rrrrrr}
			1 & 0 & 0 &  2 &    0 &   0\\
			0 & 1 & 0 &  0 & -\dfrac{5}{6} & \dfrac{1}{2}\\
			0 & 0 & 1 & -1 &  \dfrac{1}{3} &   0\\
			0 & 0 & 0 &  0 &    0 &   0
		\end{array}
	\right)
	\tilde{x} &=& 0 \\
	% 2行目
	\renewcommand{\arraystretch}{2}
	\left(
		\begin{array}{rrr}
			1 & 0 & 0\\
			0 & 1 & 0\\
			0 & 0 & 1\\
		\end{array}
	\right)
	\left(
		\begin{array}{r}
			x_1\\ x_2\\ x_3
		\end{array}
	\right)
	+
	\left(
		\begin{array}{rr}
			 2 &    0\\
			 0 & -\dfrac{5}{6}\\
			-1 &  \dfrac{1}{3}\\
		\end{array}
	\right)
	\left(
		\begin{array}{r}
			x_4\\ x_5
		\end{array}
	\right)
	&=&
	\left(
		\begin{array}{r}
			0\\ \dfrac{1}{2}\\ 0
		\end{array}
	\right)\\
	% 3行目
	\renewcommand{\arraystretch}{2}
	\left(
		\begin{array}{r}
			x_1\\ x_2\\ x_3
		\end{array}
	\right)
	&=&
	\renewcommand{\arraystretch}{2}
	\left(
		\begin{array}{rr}
			-2 &    0\\
			 0 &  \dfrac{5}{6}\\
			 1 & -\dfrac{1}{3}\\
		\end{array}
	\right)
	\left(
		\begin{array}{r}
			x_4\\ x_5
		\end{array}
	\right)
	+
	\left(
		\begin{array}{r}
			0\\ \dfrac{1}{2}\\ 0
		\end{array}
	\right)\\
\end{eqnarray*}

\end{document}