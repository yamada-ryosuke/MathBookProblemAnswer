\documentclass{jsarticle}
\usepackage{amsmath}

\begin{document}

\section{} % section1
./chapter2/inverseを使って解ける。途中経過を見たい場合は適宜Matrix.hppのMatrix::inverse()にoutput()を挿入すればよい。
\[
	\begin{array}{rrrr}
		イ) &
		\left(
			\begin{array}{rrrr}
				4 & 18 & -16 & -3\\
				0 & -1 &   1 &  1\\
				1 &  3 &  -3 &  0\\
				1 &  6 &  -5 & -1
			\end{array}
		\right) &
		ロ) &
		\left(
			\begin{array}{rrrr}
				 -3 & -1 & 1 & -1\\
				 -3 & -1 & 0 &  1\\
				 -4 & -1 & 1 &  0\\
				-10 & -3 & 1 &  1
			\end{array}
		\right)
	\end{array}
\]

\section{} % section2
行列の変形には./chapter2/ElementaryOperationを用いた。計算過程は./chapter2/2.txt参照。\\
変数を$x = (x_1,\ldots,x_n)$として、$\tilde{x}= ^{t}(x, -1)$と書くこととする。また、方程式の拡大行列を$\tilde{A}$と書くこととする。\\
イ)\\
与式$\tilde{A} \tilde{x}$を変形して、
\begin{eqnarray*}
	% 1行目
	\renewcommand{\arraystretch}{2}
	\left(
		\begin{array}{rrrrrr}
			1 & 0 & 0 &  2 &    0 &   0\\
			0 & 1 & 0 &  0 & -\dfrac{5}{6} & \dfrac{1}{2}\\
			0 & 0 & 1 & -1 &  \dfrac{1}{3} &   0\\
			0 & 0 & 0 &  0 &    0 &   0
		\end{array}
	\right)
	\tilde{x} &=& 0 \\
	% 2行目
	\renewcommand{\arraystretch}{2}
	\left(
		\begin{array}{rrr}
			1 & 0 & 0\\
			0 & 1 & 0\\
			0 & 0 & 1\\
		\end{array}
	\right)
	\left(
		\begin{array}{r}
			x_1\\ x_2\\ x_3
		\end{array}
	\right)
	+
	\left(
		\begin{array}{rr}
			 2 &    0\\
			 0 & -\dfrac{5}{6}\\
			-1 &  \dfrac{1}{3}\\
		\end{array}
	\right)
	\left(
		\begin{array}{r}
			x_4\\ x_5
		\end{array}
	\right)
	&=&
	\left(
		\begin{array}{r}
			0\\ \dfrac{1}{2}\\ 0
		\end{array}
	\right)\\
	% 3行目
	\renewcommand{\arraystretch}{2}
	\left(
		\begin{array}{r}
			x_1\\ x_2\\ x_3
		\end{array}
	\right)
	&=&
	\renewcommand{\arraystretch}{2}
	\left(
		\begin{array}{rr}
			-2 &    0\\
			 0 &  \dfrac{5}{6}\\
			 1 & -\dfrac{1}{3}\\
		\end{array}
	\right)
	\left(
		\begin{array}{r}
			x_4\\ x_5
		\end{array}
	\right)
	+
	\left(
		\begin{array}{r}
			0\\ \dfrac{1}{2}\\ 0
		\end{array}
	\right)\\
\end{eqnarray*}

\section{} % section3
3.cpp参照
\section{} % section4
$x = 0$ならば階数はn。\\
$x \neq 0$として、$y = 1 / x$と置く。\\
\begin{eqnarray*}
	\left(
		\begin{array}{cccc}
			 1 & x & \cdots & x\\
			 x & 1 & \cdots & x\\
			 \vdots & \vdots & \ddots & \vdots\\
			 x & x & \cdots & 1\\
		\end{array}
	\right)\\
	\xrightarrow{各行1/x倍}
	\left(
		\begin{array}{cccc}
			 y & 1 & \cdots & 1\\
			 1 & y & \cdots & 1\\
			 \vdots & \vdots & \ddots & \vdots\\
			 1 & 1 & \cdots & y\\
		\end{array}
	\right)\\
	\xrightarrow{(1, n)成分を要に掃き出し}
	\left(
		\begin{array}{ccccc}
			 0 & 0 & \cdots & 0 & 1\\
			 1 - y & y - 1 & \cdots & 0 & 0\\
			 \vdots & \vdots & \ddots & \vdots & \vdots\\
			 1 - y & 0 & \cdots & y - 1 & 0\\
			 1 - y^2 & 1 - y & \cdots & 1 - y & y\\
		\end{array}
	\right)\\
	\xrightarrow{2{\sim}n-1行目をn行目に、2{\sim}n-1列目を1列目に足す}
	\left(
		\begin{array}{ccccc}
			 0 & 0 & \cdots & 0 & 1\\
			 0 & y - 1 & \cdots & 0 & 0\\
			 \vdots & \vdots & \ddots & \vdots & \vdots\\
			 0 & 0 & \cdots & y - 1 & 0\\
			 1 - y^2 + (n - 2)(1 - y) & 0 & \cdots & 0 & y\\
		\end{array}
	\right)\\
	\xrightarrow{1行目をn行目から引き、1行目とn行目を入れ替える}
	\left(
		\begin{array}{ccccc}
			1 - y^2 + (n - 2)(1 - y) & 0 & \cdots & 0 & 0\\
			 0 & y - 1 & \cdots & 0 & 0\\
			 \vdots & \vdots & \ddots & \vdots & \vdots\\
			 0 & 0 & \cdots & y - 1 & 0\\
			 0 & 0 & \cdots & 0 & 1\\
		\end{array}
	\right)\\
\end{eqnarray*}
よって$y = 1$ならば階数は1。
$y \neq 1$とすると
\begin{eqnarray*}
	\xrightarrow{1{\sim}n-1行目をy-1で割る}
	\left(
		\begin{array}{ccccc}
			-n + 1 - y & 0 & \cdots & 0 & 0\\
			 0 & 1 & \cdots & 0 & 0\\
			 \vdots & \vdots & \ddots & \vdots & \vdots\\
			 0 & 0 & \cdots & 1 & 0\\
			 0 & 0 & \cdots & 0 & 1\\
		\end{array}
	\right)\\
\end{eqnarray*}
より、$y=-(n-1)$なら階数は$n-1$、$y\neq n-1$ならば階数は$n$。
まとめると、階数は$x = 1$で$1$、$x = -\dfrac{1}{n-3}$で$n-1$、それ以外で$n$。\\

\section{} % section5
\section{} % section6
イ) $A^{-1} = A^{k-1}$とすればよい。\\
ロ) $A$が正則と仮定すると、$A^2 = A$の両辺に右から$A^-1$をかけて$A=E$を得る。これは$A \neq E$に矛盾する。よって$A$は正則でない。\\
ハ) $A$が正則と仮定すると、$A^k = 0$の両辺に右から$A^-k$をかけて$E=0$を得る。これは明らかに矛盾であるから、$A$は正則でない。\\
二) $(E + A)^-1 = \sum_{i=0}^{n-1} (-A)^i$、$(E - A)^-1 = \sum_{i=0}^{n-1} A^i$とすればよい。\\

\section{} % section7
左辺のトレースを取ると、$tr(XY - YX) = tr(XY) - tr(YX) = 0$。\\
右辺のトレースを取ると、$tr(E_n) = n$。\\
よって、条件を満たす行列$X, Y$は存在しない。\\

\section{} % section8
\section{} % section9
\section{} % section10
\section{} % section11
イ)\\
$(P - E)$と$(P + E)$、及び$(P - E)$と$(P + E)^{-1}$が可換であることを確認しておく。\\
これは
\begin{equation*}
	(P - E)(P + E) = P^2 - E = (P + E)(P - E)
\end{equation*}
及び
\begin{eqnarray*}
	(P - E)(P + E)^{-1} &=& (P + E)^{-1} (P + E)(P - E)(P + E)^{-1}\\
	&=& (P + E)^{-1} (P - E)(P + E)(P + E)^{-1}\\
	&=& (P + E)^{-1} (P - E)\\
\end{eqnarray*}
によって証明される。\\
これにより、
\begin{eqnarray*}
	^t A &=& (P - E)(P + E)^{-1}\\
	&=& (^t P + ^t E)^{-1} (^t P - ^t E)\\
	&=& (P^{-1} + E)^{-1} (P^{-1} - E)\\
	&=& (P^{-1} + E)^{-1} P^{-1}P(P^{-1} - E)\\
	&=& (E + P)^{-1}(E - P)\\
	&=& -(P - E)(P + E)^{-1}\\
	&=& -A
\end{eqnarray*}
が示された。\\
ロ)\\
\begin{eqnarray*}
	E - A &=& (P + E)(P + E)^{-1} - (P - E)(P + E)^{-1}\\
	&=& 2(P + E)^{-1}\\
\end{eqnarray*}
であるから、$(E - A)^{-1} = \dfrac{1}{2}(P + E)$とすればよい。\\
ハ)\\
\begin{eqnarray*}
	A &=& (P - E)(P + E)^{-1}\\
	A(P + E) &=& P - E\\
	(A - E)P &=& -(A + E)\\
	P &=& (A + E)(E - A)^{-1}\\
\end{eqnarray*}

\section{} % section12

\end{document}