\documentclass{jsarticle}
\usepackage{amsfonts}
\usepackage{amsmath}
\usepackage{mathrsfs}

\begin{document}

\section{順序集合} % section1
\subsection{} % subsection1
定義より$a \le a$.\\
$a \le b$かつ$b \le a$を仮定する. $a \le b$より$a < b$または$a = b$であるが, $a < b$ならば$a \not> b$であるから$a \ge b$より結局$a = b$となる.\\
$a \le b$かつ$b \le c$を仮定する. $a = b$または$b = c$ならば$a \le b$は明らか. $a < b$かつ$b < c$ならば(1.5)より$a < c$が成り立つ.

\subsection{} % subsection2
\begin{eqnarray*}
	MがAの最大値
	&\Leftrightarrow& \forall{x \in A}(x \le M)\\
	&\Leftrightarrow& (\exists{x \in A}(x \le M でない))でない\\
	&\Leftrightarrow& (\exists{x \in A}(x > M))でない\\
	&\Leftrightarrow& MはAの極大
\end{eqnarray*}
最小元と極小元の一致は上の証明の双対を考えればよい.

\subsection{} % subsection3
$\mathfrak{M}$の極小元の全体は$S = \{\{x\} | x \in X\}$であることを示す. $\{x\} \in S$に対して$Y \in \mathfrak{M}$が$Y \le \{x\}$となるならば明らかに$Y = \{x\}$であるから$\{x\}$は極小. 一方$x \in \mathfrak{M} - S$ならば異なる二元$a, b \in x$が存在するので, 例えば${a} < x$であり, $x$は極小でない.

\subsection{} % subsection4
(i)(ii)を仮定する. (i)より$a$は$M$の上界である. $a$が上限でないと仮定すると$M$の上界$a'$が存在して$a' < a$が成り立つが, (ii)より$x \in M$が存在して$a' < x$を満たし, これは$a'$が上界であることに矛盾する. よって$a$は上限である.\\
$a$が上限であると仮定する. 上界であることから(i)を満たす. (ii)を満たさないと仮定すると, $a' \in M$が存在して$a' < a$かつ任意の$x \in M$に対して$x \le a'$であるが, これは$a'$が$a$より小さい$M$の上界であることを意味しており矛盾. よって(ii)を満たす.

\subsection{} % subsection5
$A = \{a_1, \cdots, a_n\}$とすれば, 上限,下限はそれぞれ$max A$, $min A$である.\\
実際$max\{ \cdots max\{a_1, a_2\}, \cdots \}, a_n \} \cdots \}$は$max A$である. $min A$は双対な証明を考えればよい.

\subsection{} % subsection6
(1.8) $(A, \le)$上の恒等写像が順序同型写像となる.\\
(1.9) 仮定より順序同型写像$f:(A, \le) \rightarrow (A', \le')$が存在するので順序同型写像$f^{-1}:(A', \le') \rightarrow (A, \le)$が存在する.\\
(1.10) 仮定より順序同型写像$f:(A, \le) \rightarrow (A', \le')$, $g:(A', \le') \rightarrow (A'', \le'')$が存在するので順序同型写像$g \circ f:(A', \le') \rightarrow (A, \le)$が存在する.\\

\subsection{} % subsection7
$x \le y$とすると, $x < y$ならば$f(x) < f(y)$なので$f(x) \le f(y)$であり, $x = y$ならば$f(x) = f(y)$なので$f(x) \le f(y)$であるから, 結局$f(x) \le f(y)$が成り立つ. ゆえに$f$は順序写像である. \\
$a < b \Rightarrow f(a) <' f(b)$の対偶をとって$f(a) <' f(b)でない \Rightarrow b \le a$が成り立つ. $f(x) \le f(y)$ならば$f(x) >' f(y)$でないので$x \le y$が成り立つ. よって$f$は順序単射である.

\subsection{} % subsection8
まず開区間$(0, 1)$から$\mathbb{R}$への順序同型写像として$f(x) = tan(\pi x - \frac{\pi}{2})$をとれるので$(0, 1) \simeq \mathbb{R}$. また開区間$(0, 1)$から開区間$(a, b)$への順序同型写像として$g(x) = a + (b - a)x$をとれるので$(0, 1) \simeq (a, b)$. よって$\mathbb{R} \simeq (a, b)$がなりたつ.

\subsection{} % subsection9
まず$\mathbb{R}$は他の3つと濃度が違うため全単射が存在しないので特に順序同型写像が存在しない.\\
次に$\mathbb{N}$から$\mathbb{Z}$や$\mathbb{Q}$への順序同型写像$f$が存在すると仮定する. $f(1) - 1 < f(1)$なので$f^{-1}(f(1) - 1) < f(1)$であるが, これは$1$が$\mathbb{N}$の最小値であることに矛盾する.\\
最後に$\mathbb{Z}$から$\mathbb{Q}$への順序同型写像$f$が存在すると仮定する. $f(0) < (f(0) + f(1)) / 2 < f(1)$なので$0 < f^{-1}(f(0) + f(1) / 2) < 1$であるが, これは明らかに矛盾する.

\subsection{} % subsection10
任意の$\chi \in C$について, 任意の$x \in X$に対して$\chi(x) \le \chi(x)$なので$\chi \le \chi$.\\
$\chi, \chi' \in C$に対して$\chi \le \chi'$ならば, 任意の$x \in X$に対して$\chi(x) \le \chi'(x)$かつ$\chi'(x) \le \chi(x)$なので$\chi(x) = \chi'(x)$であり$\chi = \chi'$.\\
$\chi, \chi', \chi'' \in C$に対して$\chi \le \chi'$かつ$\chi' \le \chi$ならば, 任意の$x \in X$に対して$\chi(x) \le \chi'(x)$かつ$\chi'(x) \le \chi'(x)$なので$\chi(x) \le \chi''(x)$であり$\chi \le \chi''$.

\section{整列集合とその比較定理} % section2
\subsection{} % subsection1
$S = \{x | x \in A, a < x\}$とする.
\begin{eqnarray*}
	bがaの直後の元
	&\Leftrightarrow& a < bかつ(a < x < bを満たすx \in Aが存在しない)\\
	&\Leftrightarrow& b \in Sかつ(x < bを満たすx \in S存在しない)\\
	&\Leftrightarrow& bはSの極小元
\end{eqnarray*}

\subsection{} % subsection2
それぞれ対偶を示す. 
$A$において降鎖$(a_n)$が存在すると仮定する. このとき集合$S = \{a_n | n \in \mathfrak{N}\}$と定義すると$S$には最小値が存在しない. 実際$a_i$が$S$の最小値と仮定すると$a_{i + 1} < a_i$かつ$a_{i + 1} \in S$であり矛盾する. ゆえに$A$は整列集合ではない.\\
$A$が整列集合でないと仮定する. このとき$A$の空でない部分集合$T$が存在して, $T$には最小値が存在しない. 写像$F:T \rightarrow \mathfrak{P}(T)$を$F(x) = \{y \in T | y < x\}$と定義すると, 選択公理により写像$F': T \rightarrow T$が存在して$F'(x) \in F(x)$が成り立つ. このとき降鎖$(a_n)$が次のように構成できる. $T$は空でないから$x \in T$がとれるので$a_0$とし, $n > 0$に対しては$a_n = F'(a_{n - 1})$とする. これは明らかに$T$における降鎖であるから$A$における降鎖でもある.

\subsection{} % subsection3
$W'$が整列集合であるため比較定理により, $W$と同型な$W'$の切片が存在しないことを証明すればよい. $W$から$W'$の切片$W'\langle a \rangle$への順序同型写像$f$が存在すると仮定する. これは$W$から$W'$への順序単射とみなせる. $W'$から$W$への包含写像を$g$とするとこれは順序単射である. よって$f \circ g$は$W'$から$W'\langle a \rangle$への順序単射であるため$(f \circ g)(a) > a$であるがこれは$(f \circ g)(a) \in W'$に反する.

\subsection{} % subsection4
$n$個の元からなる整列集合を$S$, $T = \{1, \cdots, n\}$とする. $T$の切片の濃度は$|T| - 1 = |S| - 1$以下であるから$S$と順序同型な$T$の切片は存在しない. 同様に$T$と順序同型な$S$の切片は存在しない. よって$S$と$T$は順序同型.\\
$A$を無限整列集合とする. $a_0 = min A$, $a_{n + 1} = (a_nの直後の元)$として列$(a_n)$を定義し, $S = \{a_n | n \in \mathbb{N}\}$とする. このとき$S = \mathbb{A}$ならば$a_n$を順序同型写像として$S \simeq \mathbb{N}$であり, $S \neq \mathbb{A}$ならば$S = A<sup S>$である.

\subsection{} % subsection5
双対順序集合が整列集合であることは任意の空でない部分集合は最大値をもつことと同値である. $W$が無限集合であると仮定すると, $W$または$W$の切片は$\mathbb{N}$と順序同型であるが, このような$W$の部分集合に最大値が存在することは$\mathbb{N}$に最大値が存在しないことに矛盾する. よって$W$は有限集合である.

\subsection{} % subsection6
$f$は全射なので選択公理により$g:W' \rightarrow W$が存在して$g(x) \in f^{-1}(x)$が成り立つ. $f$は順序同型写像なので$x < y \rightarrow g(x) < g(y)$が成り立つ. $W'$が整列集合でないと仮定する. すると$W'$には降鎖が存在するが, その$g$による像は$W$における降鎖となり, $W$が整列集合であることに矛盾する. よって$W'$は整列集合でない.

\subsection{} % subsection7
$x \in W$の直後の元を$succ(x)$とする. 任意の開区間$(x, succ(x))$は空でないから少なくとも一つの有理数を含む. 選択公理により写像$f:W \rightarrow \mathbb{Q}$を$f(x) \in Q \cap (x, succ(x))$が存在する. 明らかに$f$は単射なので$W$は高々可算である.

\section{Zornの補題, 整列定理} % section3$
\subsection{} % subsection1
Zornの補題$ \rightarrow $(a)\\	
$m$を$C$を満たす$X$の部分集合とする. $m$を包含し, $C$を満たす$X$の部分集合の全体を$\mathfrak{M}$とする. 示すべきは$\mathfrak{M}$の極大の存在である. $\mathfrak{M}$の任意の全順序部分集合$\mathfrak{N}$をとる. このとき$\mathfrak{N}$の上限$S = \bigcup \mathfrak{N}$は$\mathfrak{M}$の元であること, すなわち$C$を満たすことを示す. $S$の任意の有限部分集合$T$をとると, $T$の各元$t$に対して$t$を含む$\mathfrak{N}$の元をとれるので, そのうち最大のものを$A$とする. $T$は$A$の部分集合なので$C$の有限性より$C$を満たす. よって$S$の任意の部分集合が$C$を満たすことから$C$の有限性より$S$も$C$を満たすことが示された. よって$\mathfrak{M}$はZornの補題の仮定を満たすので極大をもつ. (この証明では最後以外Zornの補題を用いていないことに注意されたい.)

(a) $ \rightarrow $ (b)\\
順序集合$X$が極大な全順序集合をもつことを示す. $\phi$の任意の2元に対しても$X$の順序関係によって関係を定義するとこれは全順序集合となるので, $X$は少なくとも一つの全順序部分集合をもつ. また, $X$の部分集合$T$が全順序集合であるという条件は$T$の任意の二元が比較可能であることと同値であるから有限的である. よって(a)により$X$は極大な全順序部分集合をもつ.

(b) $ \rightarrow $ (c)\\
(b)から$A$の極大な全順序部分集合$B$をとれる. $B$の上界$s$をとる. $B$の任意の元と$s$は比較可能なので, $B \cup \{s\}$は全順序集合であるが, $B$は極大な全順序集合なので$s \in B$であるから$s$は$B$の最大値である. また$s$は$A$の極大な元である. 実際$x \ge s$を満たす$x \in A$をとると, $B \cup \{x\}$は全順序集合となるので$B$は極大な全順序集合なので$x \in B$であるから$x = s$である.

(c) $ \rightarrow $ (d)\\
(d)の仮定は$\mathfrak{M}$が包含関係による順序に関して(c)の仮定を満たすことを意味しているので$\mathfrak{M}$は極大な元を持つ.

(c) $ \rightarrow $ Zornの補題\\
上限は上界であるから明らか.

(d) $ \rightarrow $ (a)\\
$C$を満たす$X$の部分集合の全体を$\mathfrak{M}$とする. $\mathfrak{M}$の任意の全順序部分集合$\mathfrak{N}$をとる. Zornの補題$\rightarrow$(a)の証明の中で示した通り, $\mathfrak{N}$の上限は$\mathfrak{M}$の中に存在する. よって(d)の仮定を満たすので(a)は証明された.

\subsection{} % subsection2
(*)は$\mathfrak{P}(A)$の有限的な性質であることを示す. $\mathfrak{N} \subset \mathfrak{P}(A)$とする. $\mathfrak{N}$が(*)を満たすと仮定する. $\mathfrak{N}$の有限部分集合$\mathfrak{S}$を任意にとる. このとき$\mathfrak{S}$に属する任意の有限個の集合はもちろん$\mathfrak{N}$に属する有限個の集合でもあるから交わるので, $\mathfrak{S}$は(*)を満たす. 逆に$\mathfrak{N}$の任意の有限部分集合が(*)を満たすと仮定する. $\mathfrak{N}$に属する有限個の集合$M_1, \cdots, M_n \in \mathfrak{M}$をとる. $M = \{M_1, \cdots, M_n\}$は$\mathfrak{M}$の有限部分集合なので$M$に属する有限個の集合は交わるが, 特に$M_1, \cdots, M_n$自身も交わる. よって(*)は有限的な条件であるから, 定理6(a)'より(*)を満たす$\mathfrak{P}(A)$の極大な部分集合で, しかも$\mathfrak{M}$を満たすものが存在する. これを$\mathfrak{M}_0$とおく. $\mathfrak{M}_0$が(i)を満たすことは定義から明らかなので(ii)を示す. $N \in \mathfrak{P}(A)$が$\mathfrak{M}_0$の任意の元と交わるとする. このとき$\mathfrak{M}_0 \cup \{N\} \supset \mathfrak{M}_0$であるが$\mathfrak{M}_0$の極大性より$N \in \mathfrak{M}_0$である.

\subsection{} % subsection3
(i)は$C$の有限的な性質であることを示す. $B$を$\mathbb{R}$の任意の部分集合とする. $B$が(i)を満たすと仮定する. $B$の有限部分集合$B'$を任意にとる. $b_1, \cdots, b_m$を$B'$の相異なる元, $r_1, \cdots, r_m$を0でない有理数とするとき, $r_1 b_1 + \cdots + r_m b_m = 0$ならば, $b_1, \cdots, b_m$は$B$の相異なる元であるから(i)より$r_1 = \cdots = r_m = 0$が成り立つので$B'$は(i)を満たす. 逆に$B$の任意の有限部分集合が(i)を満たすと仮定する. $b_1, \cdots, b_m$を$B$の相異なる元, $r_1, \cdots, r_m$を0でない有理数とするとき, $B' = \{b_1, \cdots, b_m\}$は$B$の有限部分集合であるから(i)を満たすので, $b_1, \cdots, b_m$が$B'$の相異なる元であることから, $r_1 b_1 + \cdots + r_m b_m = 0$ならば$r_1 = \cdots = r_m = 0$が成り立つため, $B$は(i)を満たす.\\
よってTukeyの補題により(i)を満たす$\mathbb{R}$の極大な部分集合が存在する. これを$B_0$とおく. $B_0$が(ii)を満たすことを示す.\\
まず存在を背理法によって示す. $B_0$の元の線型結合によって表せない$\mathbb{R}$の元$x$が存在すると仮定する. $B_0' = B_0 \cup \{x\}$とする. $x \in B_0$と仮定すると$x = 1 \cdot x$として$B_0$の元の線型結合により表されてしまい矛盾するので, $x \notin B_0$であり$B_0' \neq B_0$である. また$B_0'$が(i)を満たすことを示す. $b_1, \cdots, b_m$を$B_0'$の相異なる元, $r_1, \cdots, r_m$を0でない有理数とし, $r_1 b_1 + \cdots + r_m b_m = 0$を満たすとする. もし$b_1, \cdots, b_m$に$x$が含まれなければ$B_0$が(i)を満たすことにより$r_1 = \cdots = r_m = 0$が成り立つ. 一方もし$b_1, \cdots, b_m$に$x$が含まれるならば, $b_1 = x$としても一般性は失われない. $r_1 \neq 0$と仮定すると$x = \left(-\dfrac{r_2}{r_1}\right) b_2 + \cdots + \left(-\dfrac{r_m}{r_1}\right) b_m = 0$となり$x$の定義に矛盾するので$r_1 = 0$. よって$r_2 b_2 + \cdots + r_m b_m = 0$であるから$r_1 = \cdots = r_m = 0$が成り立つ. よって$B_0'$は(i)を満たすが, これは$B_0$の極大性に矛盾するので$B_0$は(ii)を満たす.\\
(ii)のような表示が一意的であることを示す. $x$が$b_1, \cdots, b_k, b'_1, \cdots, b'_{k'} \in B_0$, $r_1, \cdots r_k, r'_1, \cdots, r'_{k'} \in \mathbb{Q} \backslash \{0\}$を用いて
\begin{eqnarray*}
	x
	&=& r_1 b_1 + \cdots + r_k b_k\\
	&=& r_1 b'_1 + \cdots + r_{k'} b'_{k'}
\end{eqnarray*}
の2通りで表されるとする. このとき相異なる$c_1, \cdots, c_l \in B_0$を$\{c_1, \cdots, c_l\} = \{b_1, \cdots, b_k, b'_1, \cdots, b'_{k'}\}$となるようにとる. また$s_i \in \mathbb{Q} (1 \le i \le l)$を
\begin{eqnarray*}
	s_i = \begin{cases}
		r_k & (\exists j(s_i = b_j))\\
		0 & (else)
	\end{cases}
\end{eqnarray*}
と定める. 同様にして$r'_i$に対しても$t_i$を定める. このとき
\begin{eqnarray*}
	x
	&=& s_1 c_1 + \cdots + s_l c_l\\
	&=& t_1 c_1 + \cdots + t_l c_l
\end{eqnarray*}
であるから
\begin{equation*}
	(s_1 - t_1) c_1 + \cdots + (s_l - t_l) c_l = 0
\end{equation*}
が成り立つ. ゆえに(i)より$s_1 = t_1, \cdots, s_l = t_l$が成り立つ. 今$\{c_1, \cdots, c_l\} \neq \{b_1, \cdots, b_k\}$と仮定すると$c_i \notin \{b_1, \cdots, b_k\}$がとれるが, このとき$c_i \in \{b'_1, \cdots, b'_{k'}\}$であるから$s_i = 0 < t_i$となり矛盾するため$\{c_1, \cdots, c_l\} = \{b_1, \cdots, b_k\}$. 同様に$\{c_1, \cdots, c_l\} = \{b'_1, \cdots, b'_{k'}\}$. よって$\{b_1, r_1\},\cdots, \{b_k, r_k\}$と$\{b'_1, r'_1\}, \cdots, \{b'_{k'}, r'_{k'}\}$は順番を除いて一致する.

\subsection{} % subsection4	
(a) $f, f', f'' \in \mathfrak{F}$とする. \\
$D(f) \subset D(f)$かつ$D(f)$上で$f = f$なので反射律を満たす.\\
$f \le f'$かつ$f' \le f$ならば$D(f) \subset D(f') \subset D(f)$なので$D(f) = D(f')$であり, $D(f) = D(f')$上で$f = f'$なので反対称律を満たす.\\
$f \le f'$かつ$f' \le f''$ならば$D(f) \subset D(f') \subset D(f'')$より$D(f) \subset D(f'')$であり, $f''$は$D(f')$上, 特に$D(f)$上で$f'$に等しい為$D(f)$上で$f$に等しいため推移律を満たす.\\

$S \subset \mathfrak{F}$を全順序部分集合とする. $D = \bigcup_{f \in \mathfrak{F}} D(f)$とする. 任意の$x \in D$に対して, $f \in S$が存在して$x \in D(f)$かつ任意の$f, f' \in S$に対して$x \in D(f)$かつ$x \in D(f')$ならば$f(x) = f'(x)$なので, 写像$g:A \rightarrow B$が定義出来て, $g$は任意の$f \in S$の拡大である. $g$が単射であることを示す. $x, x' \in D$が$g(x) = g(x')$を満たすとする. $f, f' \in \mathfrak{S}$が存在して$x \in D(f)$かつ$x' \in D(f')$を満たす. $S$は全順序なので$f \ge f'$としても一般性は失われない. $f(x) = g(x) = g(x') = f(x')$なので$f$の単射性より$x = x'$, すなわち$g$は単射なので$g \in \mathfrak{F}$であるから, $\mathfrak{F}$は帰納的である.

(b) $D(f_0) \neq A$かつ$V(f_0) \neq B$ならば矛盾する. 実際そのとき$x \in A\\D(f_0)$及び$y \in B\\V(f_0)$が存在するので$f_1:D(f_0) \cup \{x\} \rightarrow B$を
\begin{equation*}
	f_1(t) = \begin{cases}
		f_0(t) & (t \in A)\\
		y & (t = x)
	\end{cases}
\end{equation*}
として定義すれば, $f_1 \in \mathfrak{F}$かつ$f_0 < f_1$であり, $f_0$の極大性に矛盾する.

\section{順序数} % section4
\subsection{} %subsetction1
整列集合$A, B, C, A', B'$が$A', B', C$は互いに素で, $\mu = ord A, \nu = ord B, \rho = ord C, \nu' = ord B', \mu' = ord A'$で, ある$b \in B', a \in A'$に対して$B = B'\langle b \rangle$, $A = A'\langle a \rangle$を満たすとする.\\
(4.1) 両辺は$A \cup B \cup C$に入る順序の順序数であり, 2つの順序集合は任意の2元に対して全く同じ順序関係があるので順序同型であるから$(\mu + \nu) + \rho = \mu + (\nu + \rho)$.\\
(4.2) 包含写像$g:A \cup B \rightarrow A \cup B'$は順序単射であり, $V(g) = (A \cup B')\langle b \rangle$なので, $\mu + \nu < \mu + \nu'$.\\
(4.3) $A'' = \{x \in A' | x >= a\}$と定義すると, $A''$は空でない整列集合である. $ord A' + ord B = (ord A + ord A'') + ord B$であるから, $ord A'' + ord B \ge ord B$を示せれば(4.1)(4.2)から(4.3)が従う.\\
$ord A'' + ord B < ord B$と仮定する. このとき降鎖$(x_n)$を以下のように帰納的に定義する. $ord B > ord A'' + ord B$なので$B\langle x_0 \rangle$が$A'' + B$と順序同型となるよう$x_0 \in B$を定義できる. このとき$ord A'' + ord B\langle x_0 \rangle < ord A'' + ord B = ord B\langle x_0 \rangle$. $x_n \in B$が定義されていて$ord A'' + ord B\langle x_n \rangle < ord B\langle x_n \rangle$を満たすとき, $(B\langle x_{n} \rangle)\langle x_{n + 1} \rangle = B\langle x_{n + 1}$が$A'' + B\langle x_n \rangle$と順序同型となるよう$x_{n + 1} \in B\langle x_n \rangle$を定義すると, $ord A'' + ord B\langle x_{n + 1} \rangle < ord A'' + ord B\langle x_{n} \rangle = ord B\langle x_{n + 1}\rangle$が成り立つ. よって$B$は降鎖を持つが, これは$B$が整列集合であることに矛盾する.\\
よって$ord A'' + ord B \ge ord B$が成り立つ.

\subsection{} %subsetction2
$\mu < \nu$を仮定する. 整列集合$A$及び$a \in A$を$ord A = \nu$, $ord A\langle a\rangle = \mu$となるようにとれる. このとき$B = \{x \in A | x \ge a\}$と定義すると, $A\langle a \rangle  + B = A$かつ$B \neq \phi$なので$\rho = ord B$とすると$\nu = \mu + \rho$かつ$\rho > 0$.\\
$\nu = \mu + \rho$かつ$\rho > 0$を満たすような順序数$\rho$が存在すると仮定する. 互いに素な集合$A, B$を$ord A = \mu, ord B = \rho$と満たすようにとると, $\nu = ord (A + B)$である. $A = (A + B)\langle min B \rangle$であるから$\mu < \nu$.

\subsection{} %subsetction3
$\mu + \nu < \mu + \nu' \Rightarrow \nu < \nu'$の対偶$\nu' \le \nu \Rightarrow \mu + \nu' \le \mu + \nu$を示す. $\nu' = \nu$ならば明らかに$\mu + \nu' = \mu + \nu$であるし, $\nu' < \nu$ならば(4.2)より$\mu + \nu' < \mu + \nu$が成り立つ.\\
$\mu + \nu = \mu + \nu' \Rightarrow \nu = \nu'$を示すためには対偶$\nu' \neq \nu \Rightarrow \mu + \nu' \neq \mu + \nu$を示せばよいが, これは順序数が全順序であることと(4.2)から明らかである.\\
$\mu + \nu = \mu' + \nu \Rightarrow \mu = \mu'$は必ずしも成り立たない. 実際$\mu = 0, \mu' = 1, \nu = \omega$の場合が, $\mu + \nu = \mu' + \nu = \omega$かつ$\mu \neq \mu'$であり反例となる.

\subsection{} %subsetction4
集合$A, B, C, A', B'$を$A', B', C$が互いに素かつ$ord A = \mu, ord B = \nu, ord C = \rho, ord A' = \mu', ord B' = \nu'$かつ$A'\langle a \rangle = A, B'\langle b \rangle = B$となるように定義する.\\
(4.4) 両辺は$A \times B \times C$に入る整列順序の順序数であり, 2つの整列集合は任意の2元に対して全く同じ順序関係があるので順序同型であるから$(\mu\nu)\rho=\mu(\nu\rho)$が成り立つ.\\
(4.5) $A \times B = (A \times B')\langle (min A, b) \rangle$であるから$\mu\nu < \mu\nu'$.\\
(4.6) $\mu\nu > \mu'\nu$と仮定する. このとき次のように帰納的に降鎖$(x_n)$を定義できる. $ord (A \times B)\langle x_0\rangle = ord A' \times B$となるように$x_0 \in A \times B$を定義できる. このとき$ord (A' \times B)\langle x_0 \rangle < ord A' \times B = ord (A \times B)\langle x_0 \rangle$. $x_n \in A \times B$が定義されていて$ord (A' \times B)\langle x_n \rangle < ord (A \times B)\langle x_n \rangle$を満たすとき, $((A \times B)\langle x_{n} \rangle)\langle x_{n + 1} \rangle = (A \times B)\langle x_{n + 1}\rangle$が$(A' \times B)\langle x_n \rangle$と順序同型となるよう$x_{n + 1} \in (A \times B)\langle x_n \rangle$を定義すると, $ord (A' \times B)\langle x_{n + 1} \rangle < ord (A'\times B)\langle x_{n} \rangle = ord (A \times B)\langle x_{n + 1}\rangle$が成り立つ. よって$A \times B$は降鎖を持つが, これは$A \times B$が整列集合であることに矛盾する.\\
よって$ord A ord B \le ord A' ord B$が成り立つ.

\subsection{} %subsetction5
任意に$n \in \mathbb{N}$をとる. 整列集合$A = {0, \cdots, n - 1} \times \mathbb{N}$を定義すると$ord A = n\omega$. 写像$f:\mathbb{N} \rightarrow A$を$f(i, j) = jn + i$と定義すると$f$は順序同型写像なので$\omega = n\omega$.

\subsection{} %subsetction6
$\mu\nu < \mu\nu' \Rightarrow \nu < \nu'$の対偶$\nu \ge \nu' \Rightarrow \mu\nu \ge \mu\nu'$を示す. $\nu = \nu'$または$\mu = 0$ならば明らかに$\mu\nu = \mu\nu'$であるし, $\mu > 0$かつ$\nu > \nu'$ならば(4.5)より$\mu\nu > \mu\nu'$である.\\
$\mu\nu = \mu\nu', 0 < \mu \Rightarrow \nu = \nu'$を示すためにはと同値な命題$\nu \neq \nu', 0 < \mu \Rightarrow \mu\nu \neq \mu\nu'$を示せばよいが, これは順序数が全順序なことと(4.5)\\
$\mu\nu = \mu'\nu, 0 < \mu \Rightarrow \mu = \mu'$は必ずしも成り立たない. 実際$\mu = 2, \nu = \omega$が前問より反例となる.

\subsection{} %subsetction7
集合$A, B, C$を$A$と$B$が互いに素かつ$ord A = \mu, ord B = \nu, ord C = \rho$となるようにとる. (4.7)の両辺は$C \times (A \cup B) = (C \times A) \cup (C \times B)$に入る順序の順序数であり, 2つの順序集合は任意の2元に対して全く同じ順序関係があるので順序同型であるから$\rho(\mu + \nu) = \rho^mu + \rho\nu$.\\
$(\mu + \nu)\rho = \mu\rho + \nu\rho$は$\mu = \nu = 1, \rho = \omega$の場合が前問より反例となる.

\subsection{} %subsetction8
\begin{eqnarray*}
	\mu が孤立数
	&\Leftrightarrow& 直前の順序数\nu が存在して, かつ\nu = ord A\langle M \rangle を満たすM \in Aが存在する.\\
	&\Leftrightarrow& M \in Aが存在して, \rho < \mu を満たす任意の順序数\rho に対して, \rho \le ord A\langle M \rangle を満たす.\\
	&\Leftrightarrow& M \in Aが存在して, 任意のx \in Aに対して, A\langle x \rangle \le \langle M \rangle を満たす.\\
	&\Leftrightarrow& M \in Aが存在して, 任意のx \in Aに対して, x \le Mを満たす.\\
	&\Leftrightarrow& Aには最大値が存在する.
\end{eqnarray*}

\subsection{} %subsetction9
互いに素な集合$A, B$を$ord A = \mu, ord B = \nu$を満たすようにとる.\\
前問より
\begin{eqnarray*}
	\nu が孤立数
	&\Leftrightarrow& Aに最大値Mが存在\\
	&\Leftrightarrow& A + Bに最大値Mが存在\\
	&\Leftrightarrow& \mu + \nu が孤立数
\end{eqnarray*}
また
\begin{eqnarray*}
	\mu および \nu がともに孤立数
	&\Leftrightarrow& A, Bにそれぞれ最大値M_0, M_1が存在\\
	&\Leftrightarrow& A \times Bに最大値(M_0, M_1)が存在\\
	&\Leftrightarrow& \mu \nu が孤立数
\end{eqnarray*}

\subsection{} %subsetction10
$n \in \mathbb{N}$, $(a_\alpha)_{\alpha \in \Lambda} \in A$に対して, $a_\alpha \neq e_\alpha$を満たす$\alpha \in \Lambda$のうち$n$番目に大きいものが存在すれば, それを$f_n((a_\alpha)_{\alpha \in \Lambda})$と書くこととする. また, $x \in A$に対して$f_n(x)$が存在することを$P(n)$と書くこととする.\\
最小値の存在しない空でない部分集合$B \subset A$が存在すると仮定する. このとき$\Lambda$の降鎖$(x_n)$及び$B$の部分集合列$(S_n)$を次のように帰納的に定義できる.\\
$S_0 = B$と定義する. $S_0$には最小限が存在しないので各元は$P(1)$を満たす.\\
$S_n$が定義されていて, $S_n$は$B$または$B$の切片であり, $S_n$の各元は$P(n + 1)$を満たすと仮定する. $S_n$の各元$t$に対して$f_{n + 1}(t)$を対応させる写像の像は$\Lambda$の空でない部分集合であるから, その最小値を$x_n$と定義する. $f_{n + 1}(t) = x_n$となる$t \in \S_n$の全体を$S'$とすると, $(a_\alpha)_{\alpha \in \Lambda} \in S'$に対して$a_{x_n} \in A_{x_n}$を対応させる写像の像は$A_{x_n}$の空でない部分集合であるから, 最小値を$m$とする. この写像の$\{m\}$の逆像を$S_{n + 1}$と定義する. $S_{n + 1}$は明らかに$B$または$B$の切片である. また$P(n + 2)$を満たさない$S_{n + 1}$の元が存在すると仮定すると, それは$S_{n + 1}$の最小限となり矛盾するので, $S_{n + 1}$の各元は$P(n + 2)$を満たす.\\
よって$\Lambda$には降鎖が存在するが, これは$\Lambda$が整列集合であることに矛盾するので, $A$は整列集合である.

\subsection{} %subsetction11
互いに素な整列集合$A, B, C$をそれぞれ順序数が$\mu, \nu, \rho$となるように定める.\\
$\mu^\nu \mu^\rho = \mu^{\nu + \rho}$の両辺は$A^{B + C}$に入る整列順序の順序数であり, 全く同じ順序であるから両辺は等しい.\\
写像$F:A^{B \times C} \rightarrow (A^B)^C$を$F(f)(a)(b) = f(a, b)$と定義するとこれは順序同型なので$(\mu^\nu)^\rho = \mu^{\nu\rho}$が成り立つ.

\end{document}